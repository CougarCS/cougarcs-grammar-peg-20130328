%% vim:tw=66:spell:wrap:ft=tex
\documentclass[%
        hyperref={%
                pdfauthor={Zakariyya Mughal},%
                pdfpagemode={None},pdfpagelayout={SinglePage}}%
        xcolor={x11names},%
]{beamer}
\usetheme{Warsaw}
\usecolortheme{crane}
\usepackage{textcomp}
\usepackage{fancyvrb}
\usepackage{changepage}
\usepackage{framed}
\usepackage{multicol}
\usepackage{wasysym}
\usepackage{listings}
\lstset{basicstyle=\ttfamily,
numbers=left,
escapeinside=||
}
\usepackage[T1]{fontenc}
\newenvironment{indented}{\begin{adjustwidth}{1.5em}{}}{\end{adjustwidth}}

\usepackage{tikz}
\usetikzlibrary{snakes,arrows,shapes,automata}

\title[PEG]{Parsing expression grammar}
\author{Zaki Mughal}
\institute{University of Houston:\\CougarCS}
\date{2013 Mar 28}
\begin{document}

\frame{\titlepage}

\begin{frame}
	\textbf{Parsing expression grammars} can be used to match text.
	\\\pause\qquad So can \textbf{regular expressions}.
	\\\pause\qquad\qquad Why learn both?
\end{frame}

\begin{frame}[fragile]
	Regular expressions can't match:\\
		\pause 1 + 2 * ( 1 + 4 )
		\pause \lstinputlisting[numbers=none,tabsize=2,language=html]{inc/example.html}
		\pause \lstinputlisting[numbers=none,tabsize=2,language=c]{inc/example.c}
\end{frame}

\begin{frame}
	\begin{center} \Huge We need a grammar. \end{center}
\end{frame}

\begin{frame}
	\begin{center} \Huge Grammars are recursive. \end{center}
\end{frame}

\begin{frame}
	Here's part of a grammar:
	\begin{framed}
		\lstinputlisting[basicstyle=\tiny]{inc/java.ebnf}
	\end{framed}
	This is from the \href{http://docs.oracle.com/javase/specs/jls/se7/html/jls-18.html}{Java Language Specification}
\end{frame}

\begin{frame}
	A single production rule:
	\lstinputlisting{inc/java-explain.ebnf}
\end{frame}
\begin{frame}
	A single production rule:
	\lstinputlisting{inc/java-explain-more.ebnf}
\end{frame}

\begin{frame}
	\Huge But there's a problem \\
	\pause \qquad when you have alternatives \\
	\pause \qquad \qquad you have ambiguity
	\pause Which alternative to follow --- multiple trees
\end{frame}

\begin{frame}
    \Huge
    Let's eat Grandma! \\
    \pause\qquad uh\ldots \\
    \pause Let's eat, Grandma! \\
    \pause\qquad *wipes brow* \\
\end{frame}

\begin{frame}
	\begin{itemize}
		\item Ambiguity happens often with (context-free)
			grammars\footnote{These are part of the Chomsky hierarchy
			along with regular expressions.}.
		\pause\item Alternatives: leftmost, rightmost
	\end{itemize}
\end{frame}

\begin{frame}
	\begin{itemize}
		\item PEG is simpler.
		\pause\item It follows the first alternative that
			matches.
	\end{itemize}
\end{frame}

\begin{frame}
	\begin{center}
		\Huge
		Let's write a calculator!
	\end{center}
	\begin{itemize}
		\item Operation: +- */ (with precedence)
		\item parentheses group operations
		\item variable assignment (a -- z)
	\end{itemize}
\end{frame}

\begin{frame}
	Future:
	\begin{itemize}
		\item longer variable names
		\item negative sign (-1)
		\item decimal numbers (1.25)
		\item functions (\(\ln\)), constants (\(\pi\))
		\item implicit multiplication (\(2a\))
	\end{itemize}
\end{frame}

\begin{frame}
	\begin{itemize}
		\item \href{http://bford.info/packrat/}{The Packrat Parsing and Parsing Expression Grammars Page}
		\item \href{http://piumarta.com/software/peg/}{peg/leg (C)}
		\item \href{http://pyparsing.wikispaces.com/}{pyparsing (Python)}
		\item \href{http://pegex.org/}{Pegex (Perl)}
	\end{itemize}
\end{frame}


\end{document}
